\documentclass[11pt]{letter}
\usepackage{setspace,amsfonts,amsmath,fullpage,color}

\makeatletter
\let\@texttop\relax
\makeatother

\setlength{\textheight}{9in}

\definecolor{DarkBlue}{rgb}{0,0,.8}

\newcommand\tcb{\textcolor{DarkBlue}}
\newcommand\tcr{\textcolor{red}}

\begin{document}

\signature{Yuanyuan Feng,\\Lei Li, \\ Jian-Guo Liu, \\ Xiaoqian Xu}

\date{\hfill}
\longindentation=0pt                       % needed to get closing flush left
 
\begin{letter}{
}

\opening{\vspace{-.7in}
{\Large \textbf{Response to Referee}}}
\vspace{.2in}

\tcb{We are very grateful to the referees for their time and effort in the review of our manuscript. }

\textit{\textbf{Referee}}

\tcb{Below, we address the comments point by point .}

The authors have improved this manuscript. However, there are still
some problems that need to be improved. Firstly, the authors have reproved Lemma 3.9, but there are still some doubts in the proof, e.g., how to ensure that $u(t)\ge u_0$ and $f$ is positive,

\tcb{We should not have skipped the steps. We have added several sentences to clarify this. }

 and the equation $u_0+\frac{1}{\Gamma(\gamma)}\int_{t_2-t_1}^{t_2}(t_2-s)^{\gamma-1}f(u^0)ds=u^1(t_1)$ is not correct. 
 
 \tcb{What we had was misleading. By a change of variables, $u_0+\frac{1}{\Gamma(\gamma)}\int_{t_2-t_1}^{t_2}(t_2-s)^{\gamma-1}f(u^0)ds=u_0+\frac{1}{\Gamma(\gamma)}\int_0^{t_1}(t_1-\tau)^{\gamma-1}f(u^0)d\tau=u^1(t_1)$.}
 
 Secondly, in Proposition 5.8, the
authors add the scope of $p$, but according to the proof this range seems unreasonable.

\tcb{We added several more sentences in the proof to justify this range. To understand the range of $p$, one can consider the ODE $u_t=-u^p$, for which the claim of this proposition is always true for any $p\in \mathbb{R}$. Here the only condition we need for $f(u)=Au^p$ ($A<0$) is local Lipschitz on $(0,\infty)$, which is true for all $p\in\mathbb{R}$.}

\addtolength{\medskipamount}{-.8\medskipamount}
\closing{Sincerely,}

\newdimen\indentedwidth
\indentedwidth=\textwidth
\advance\indentedwidth -\longindentation

\end{letter}


\end{document}
